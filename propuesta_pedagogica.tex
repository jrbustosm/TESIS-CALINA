% Autor: Jose Ricardo Bustos Molina
%        Universidad del Tolima
%        jrbustosm@ut.edu.co
%

\section{Propuesta pedagógica gamificada}

La aplicación de una estrategia que use gamificación en un currículo escolar es un paso importante hacia una 
forma innovadora de enseñar. Dado que la metodología gamificada es un cambio de paradigma en la forma en que 
muchos educadores ven la enseñanza, puede ser difícil de aceptar e implementar, y la resistencia a vencer 
puede ser tan grande que termina por no iniciarse la gamificación o incluso peor, usarse inadecuadamente. Por 
lo que el desafío de integrar la gamificación en currículo es un proceso complejo que puede incluir la 
modificación del mismo plan de estudios, los programas, metodologías y procesos, e incluso mejorar los 
recursos físicos de la institución tanto humanos, como académicos y físicos, para lograr implementar desde los 
cimientos que los elementos de juegos promuevan los objetivos del proceso pedagógico y la evaluación que 
conlleva.

Bajo este esquema la gamificación sirve como el marco esquelético del plan de estudios el cual consiste en 
aplicar aspectos similares a los de se usan en videojuegos sin poner en peligro el rigor del mismo plan de 
estudios, ni siendo un reemplazo o su objetivo final en lugar del verdadero objetivo. Es a través de la 
utilización de incentivos dirigidos que se logra desarrollar una motivación intrínseca la cual busca aumentar 
la producción e interés en los estudiantes, que mediante los métodos gamificados es aprovechada la mentalidad 
que surge de los juegos en los estudiantes, logrando así mantener los aspectos importantes de diversión 
presentes en el trasfondo.

A continuación se hace la propuesta siguiendo el modelo aportado por \citeA{Tyler2013-in},

\subsection{¿Qué fines desea alcanzar la institución o el docente?}

Inicialmente se debe replantear la misión y visión del docente respecto al proceso educativo para con sus 
estudiantes, ya que si se utilizan elementos de la gamificación de forma descontextualizada se pierde el 
verdadero sentido de generar una experiencia de aprendizaje relevante, esto es que el docente deben replantear 
los objetivos con base a los propios objetivos de los educandos, para que exista inicialmente una motivación 
intrínseca por parte de ellos, y no que los estudiantes sigan objetivos que a lo mejor no les interesa.

Por lo tanto, es importante que los docentes que van a implementar un currículo con estrategias gamificadas, 
reflexionen sobre los intereses, capacidades y necesidades de sus estudiantes, y se replanteen desde el inicio 
dónde quieren que estén al final de su juego (plan de estudios). Por lo tanto, al realizar esta identificación 
de necesidades y motivaciones será más fácil y apropiado para los educadores determinar cómo se debe 
configurar su marco gamificado dando un contexto dirigido.

\subsection{¿Cuáles experiencias educativas ofrecen mayores posibilidades para alcanzar esos fines?}

En el caso del uso de técnicas de gamificación se puede caer en el error de hacerlo solamente para divertir, 
para motivar, para hacer algo diferente, o porque está de moda, pero sin un propósito educativo claro y 
conectado con el currículum o a los mismos estudiantes, no es tan útil. Siendo esencial construir los planes 
de estudio, con el fin de enseñar conocimientos actualizados y útiles para que los estudiantes se sientan 
identificados con su propio proceso, creando así una flexibilidad pudiendo aplicar estos conocimientos en 
diversas situaciones en su vida cotidiana.

Los estudiantes es más que probable que estén interesados en un plan de estudios donde los maestros crean una 
historia o siguen un tema en particular en el cual se sientan involucrados. Por lo que ``tematizar'' el 
currículo desarrolla un contexto interesante para que el contenido del curso sea percibido de mejor manera por 
los estudiantes. Por último, es importante señalar que la implementación de tecnologías avanzadas (incluyendo 
la aplicación CALINA) no es un requisito para implementar una gamificación académica exitosa.

\subsection{¿Cómo organizar eficazmente esas experiencias?}

La gamificación incorpora elementos clásicos del diseño de vídeo juegos y juegos tradicionales, estos 
elementos los podemos discriminar en diégesis, mecánicas, dinámicas, estéticas y emociones como aspectos 
interdependientes, \citeA{CECHELLA2018}. En el mundo de los videojuegos, existen algunos juegos muy complejos 
con argumentos, temas y giros de trama muy complicados que ofrecen una gran cantidad de incentivos para una 
participación exitosa de los jugadores. Sin embargo, tales modelos pueden no ser ideales para crear un marco 
curricular gamificado, y para este caso mantenerlo simplificado y bien definido es una mejor opción.

Por lo que es importante considerar el tiempo y el esfuerzo necesarios, no solo para construir el marco 
curricular, sino también para mantenerlo. Si el sistema gamificado es demasiado complejo, se convierte en una 
carga tanto para el docente como para el alumno. Si el juego es demasiado complejo, los estudiantes se 
confundirán y pueden perder interés desde el principio. Por lo que las mecánicas, diégesis y estéticas 
seleccionadas deben ser manejables, y no llevar demasiado tiempo comprenderlas, es así como la gamificación 
mejora o complementa el plan de estudios, sin robar el protagonismo a los contenidos de las asignaturas como 
tal.

\subsection{¿Cómo comprobar si se han alcanzado los fines propuestos?}

En un marco curricular gamificado comienza con una simple recompensa que conduce al estudiante a continuar 
participando, por lo que las recompensas aumentarían exponencialmente en valor a medida que avance e incluso
se pueden volver más difíciles de alcanzar a medida que avanza el plan de estudios, así como sucede 
regularmente en la vida cotidiana.

Sin embargo no se debe perder la perspectiva de los juegos, y mas que asignar puntos de experiencia, en este 
tipo de esquema gamificado se debe tender a ofrecer la mayor cantidad posible de retroalimentación al 
estudiante, que al igual que en los videojuegos sepa en todo momento dónde ubicarse con respecto a su progreso 
en el plan de estudios.

\subsection{Estrategia gamificada propuesta para esta investigación}

Los marcos gamificados solo motivan a los educandos cuando estos se juegan por voluntad propia y con la 
intensidad requerida. Si no se hace apropiación de las mecánicas del plan de estudios o no se juega 
regularmente, los estudiante se olvidarán de él y perderán el contacto con la motivación inicial que el 
sistema gamificado plantea en primer lugar, no logrando generar las dinámicas deseadas dentro del plan de 
estudios. Por lo tanto, una vez que el docente haya desarrollado su marco gamificado con temas, un vocabulario 
asociado e incentivos, deberían tener explicaciones claras del sistema gamificado, de cómo los estudiantes 
pueden ganar recompensas a lo largo de su plan de estudios y tener una forma para que estos puedan hacer un 
seguimiento de sus propios progresos a lo largo del curso.

Por otro lado, el docente puede encontrar que es requerido cambiar las mecánicas, diégesis o estéticas 
planteadas para abordar las necesidades de los estudiantes bajo un contexto concreto. Al ser flexibles a los 
cambios, es más probable que el docente aborde las necesidades académicas de sus estudiantes y logre mantener 
el interés con el plan de estudios gamificado.

Con los conceptos anteriores planteados y usando los 11 principios descritos en la Tabla \ref{tab:principios}, 
se plantea el uso de una estrategia gamificada para la guía mostrada en el Anexo \ref{anexo:guia}, la cual es 
una guía pedagógica de la asignatura de Inglés del grado noveno en el colegio Ciudad Luz de Ibagué ``GUÍA 
INGLÉS – TERCER PERIODO-2022'', para su desarrollo se propone un diseño por capas, la razón principal de este 
tipo de diseño es cumplir el principio 1, donde es la estrategia gamificada la que debe acondicionarse al tema 
de la clase y no al contrario, al realizar el diseño de esta manera, las intenciones de la guía pedagógica 
quedan intactas, y la gamificación es un complemento que enriquece la guía como tal, mas no la reemplaza.

Se plantean 4 capas, la capa 0 es la más profunda y la que sostiene el proceso pedagógico, es la guía en sí, 
la capa 1 es la capa de recompensas que es la encargada de motivar o ayudar el desarrollo y culminación de la 
capa 0, mediante incentivos propios de la gamificación en preferencia con una retroalimentación rápida o con 
recompensas de uso práctico por parte de los estudiantes, la capa 2 es la capa de narrativa que ayuda a 
entrelazar las diferentes recompensas entregadas en la capa 1, creando una diégesis que hacen mas deseables la 
obtención de recompensas de la capa 1, por último una tercera capa que enriquece todo el proceso dando retos y 
extras a los estudiantes para premiar su esfuerzo e incentivar su curiosidad por la continuación del 
aprendizaje en sí.

\begin{figure}[ht]
\caption{Diseño de estrategia gamificada, modelo por capas}
\label{img:tflow}
\centering
\colorlet{lightgray}{black!30}
\begin{tikzpicture}
	\draw[pattern=dots,pattern color=lightgray] (0,0) -- (4*1.2/6,1.2) --({(12-1.2)*4/6},1.2) -- (8,0) -- (0,0);
	\draw[pattern=grid,pattern color=lightgray] (4*1.2/6,1.2) -- (4*2.4/6,2.4) --({(12-2.4)*4/6},2.4) -- ({(12-1.2)*4/6},1.2) -- (4*1.2/6,1.2);
	\draw[pattern=north east lines,pattern color=lightgray] (4*2.4/6,2.4) -- (4*3.6/6,3.6) --({(12-3.6)*4/6},3.6) -- ({(12-2.4)*4/6},2.4) -- (4*2.4/6,2.4);
	\draw[pattern=crosshatch,pattern color=lightgray] (4*3.6/6,3.6) -- (4*4.8/6,4.8) --({(12-4.8)*4/6},4.8) -- ({(12-3.6)*4/6},3.6) -- (4*3.6/6,3.6);
	\node at (4,0.6) {Capa 0};
	\node at (4,1.8) {Capa 1};
	\node at (4,3) {Capa 2};
	\node at (4,4.2) {Capa 3};
	\draw[-triangle 90] (9,0.6) -- (8,0.6);
	\draw[-triangle 90] (9,1.8) -- (8,1.8);
	\draw[-triangle 90] (9,3) -- (8,3);
	\draw[-triangle 90] (9,4.2) -- (8,4.2);
	\node[anchor=west] at (9.1, 0.6) {Guía de aprendizaje};
	\node[anchor=west] at (9.1, 1.8) {Motivación y recompensa};
	\node[anchor=west] at (9.1, 3) {Diégesis y narración};
	\node[anchor=west] at (9.1, 4.2) {Retos y extras};
\end{tikzpicture}

\\
{\footnotesize Fuente: de elaboración propia}
\end{figure}

\subsubsection{Objetivo de aprendizaje de la guía pedagógica}

La guía de aprendizaje busca que los estudiantes puedan argumentar sobre gente común que cambia el mundo y 
logren describir acciones heroicas usando el pasado simple en Inglés. Adicional, los estudiantes puedan 
escribir sobre las cosas que puede o no hacer en el idioma Inglés, así como las que podía hacer manejando un 
tiempo pasado.

\subsubsection{Capa 1. Motivación y recompensas}

Esta capa busca la selección de mecánicas que fortalezcan la guía a trabajar permitiendo su incentivar en los 
educandos su desarrollo y culminación, para esto vamos a usar la aplicación CALINA, a continuación se 
mencionan los objetos virtuales y estrategias que se emplean, y su razón,

\uline{Dinero Virtual}: Se crea una misma moneda virtual para cada uno de los dos grupos que van a implementar 
la estrategia gamificada, en ambos grupos G\textsubscript{2} y G\textsubscript{3} se selecciona el símbolo de 
moneda \adjincludegraphics[height=1em,valign=c]{guarani} debido a que el nombre de la docente empieza con la 
letra G y es una forma que ellos se identifiquen y acepten la moneda mas fácilmente, la moneda virtual se 
escoge como eje central de la estrategia y sirve como indicador de puntos de experiencia logrados por los 
estudiantes, eso quiere decir que los demás objetos virtuales creados para la estrategia (al menos la mayoría) 
se deben poder conseguir a través del uso de esta divisa la cual es conseguida durante el desarrollo de la 
guía de aprendizaje, o bien, los objetos virtuales pueden ayudarnos a conseguir mas divisas, para así, 
poder conseguir otros objetos que representen un valor mayor para el estudiante por su utilidad o significado.

\uline{Formas de conseguir dinero virtual}, a continuación se listan las formas en las que se seleccionan para 
que el estudiante consiga estos puntos de experiencia

\begin{itemize}
	\item 200 puntos cada día por asistencia a clase.
	\item 10 puntos cada día que use la aplicación CALINA.
	\item 100 puntos por tener una carta de participación en el \textit{spelling bee}.
	\item 500 puntos por tener una carta de ganador en el \textit{spelling bee}.
	\item 50 puntos por contestar correctamente la trivia diaria.
	\item 50 puntos por entregar una de las actividades de la guía.
	\item 500 puntos por contestar correctamente las actividades de la semana.
	\item Puntos otorgados por el docente, debido a su criterio, bien sea por transferencia directa de 
		divisa o una crear una carta que tenga el \textit{trigger} para aumentar el dinero.
	\item Transferidos por otro estudiante
\end{itemize}

\uline{Formas de gastar dinero virtual}, para esta sección se seleccionan mecánicas bajo el concepto que la 
adquisición de puntos de experiencia se vuelva significativo para el estudiante, y el objetivo es que para 
el estudiante los 100 o 500 puntos que consigue signifique un valor intrínseco el cual se puede traducir en un 
beneficio gracias a su esfuerzo o tiempo de dedicación

\begin{itemize}
	\item Compra de tarjeta de un catálogo de cartas publicada por el docente
	\item Puntos restados por el docente, debido a su criterio, por transferencia de una carta que tenga 
		el \textit{trigger} para disminuir el dinero.
	\item Transferir dinero a otro estudiante
\end{itemize}

\uline{Cartas implementadas para el desarrollo de la estrategia}, estas cartas las implementa el docente para 
complementar el desarrollo normal de la guía,

\begin{tcolorbox}[colback=red!5!white,colframe=red!75!black,title=0000 - ENGLISH 9th GRADE]
\begin{tabular}{ p{30mm} p{117mm}}
\adjincludegraphics[width=30mm,valign=t]{CALINA/openclipart/item224}
&
\textbf{Tipo:} Agrupación\newline
\textbf{Dificultad:} Muy Fácil\newline
\textbf{Descripción:} Group created for the ninth grade class of the Ciudad Luz public school. Come and enjoy 
learning English\newline
\textbf{\textit{Trigger:}} \verb/Z_T_M:10/
\textbf{Dinero inicial:} 0\newline
\end{tabular}
\tcblower
Es la carta de agrupación obligatoria, y se va a usar para ambos grupos de prueba G\textsubscript{2} y 
G\textsubscript{3}, debido a que se quiere ver si surgen dinámicas entre los dos grupos, por lo que su 
propósito es la de agrupar a todos los estudiantes de grado noveno de la institución, y así poder transferir 
cartas entre sus integrantes. Por otro lado, posee un \textit{trigger} que indica que cada día el estudiante 
tiene derecho a 10 puntos solo por ingresar a la aplicación, se selecciona un nivel bajo ya que el entrar a 
la aplicación todos los días no es parte fundamental de la guía pedagógica. Inicialmente los estudiantes
inician con ningún punto.
\end{tcolorbox}

Las siguientes cartas se van a usar en el grupo G\textsubscript{2}, ya que las tarjetas para este grupo no 
tienen una narrativa y solo se busca informar sobre su utilidad.

\begin{tcolorbox}[colback=green!5!white,colframe=green!75!black,title=0001 - Attendance Card]
\begin{tabular}{ p{30mm} p{117mm}}
\adjincludegraphics[width=30mm,valign=t]{CALINA/simbolo_1}
&
\textbf{Tipo:} Recompensa\newline
\textbf{Dificultad:} Fácil\newline
\textbf{Descripción:} Receiving this card, you get 200 in cash.\newline
\textbf{\textit{Trigger:}} \verb/R_T_M:200/\newline
\textbf{Fecha caducidad:} 2-7 días luego de la instrucción, a criterio del docente
\end{tabular}
\tcblower
Es la carta que se va a usar para asignar 200 puntos si se desea por parte del estudiante a raíz de su 
asistencia a clase, su propósito es estimular en los estudiantes la puntualidad y su asistencia a la clase, es 
una tarjeta que la docente va a poner disponible a la entrada del salón en los primeros minutos y se retira al 
final de la clase o a criterio del docente.
\end{tcolorbox}

\begin{tcolorbox}[colback=blue!5!white,colframe=blue!75!black,title=0002 a 0005 - Trivia Card]
\begin{tabular}{ p{30mm} p{117mm}}
\adjincludegraphics[width=30mm,valign=t]{CALINA/simbolo_2}
&
\textbf{Tipo:} Informativa\newline
\textbf{Dificultad:} Normal\newline
\textbf{Descripción:} Answer correctly, what is the past tense of XXX?, you receive 50 in cash.\newline
\textbf{\textit{Trigger:}} \verb/E_W:YYY_M:50#E#G:Respuesta Correcta/\newline
\textbf{Fecha caducidad:} 2-7 días luego de la instrucción, a criterio del docente
\end{tabular}
\tcblower
Estas son cuatro cartas que el docente pone a disposición al inicio o final de clase, son preguntas sobre la 
conjugación en pasado de cuatro verbos, si el estudiante contesta correctamente cada trivia le da 50 
puntos de experiencia por cada una, en total puede ganar 200 puntos en cada clase, el propósito de esta 
tarjeta es incentivar el aprendizaje fuera de clase y la investigación
\end{tcolorbox}

\begin{tcolorbox}[colback=green!5!white,colframe=green!75!black,title=0006 - Spelling Bee]
\begin{tabular}{ p{30mm} p{117mm}}
\adjincludegraphics[width=30mm,valign=t]{CALINA/simbolo_3}
&
\textbf{Tipo:} Recompensa\newline
\textbf{Dificultad:} Normal\newline
\textbf{Descripción:} Reward for participating in the spelling bee contest, you receive 100 in cash.\newline
\textbf{\textit{Trigger:}} \verb/R_T_M:100/\newline
\textbf{No es transferible por parte de los estudiantes}\newline
\textbf{No se debe imprimir código}
\end{tabular}
\tcblower
Da 100 puntos por participar en el concurso de deletreo que se hace en clase, su finalidad es motivar a los 
estudiantes a competir en este concurso, ya que es opcional su participación.
\end{tcolorbox}

\begin{tcolorbox}[colback=green!5!white,colframe=green!75!black,title=0007 - Spelling Bee finalist]
\begin{tabular}{ p{30mm} p{117mm}}
\adjincludegraphics[width=30mm,valign=t]{CALINA/simbolo_4}
&
\textbf{Tipo:} Recompensa\newline
\textbf{Dificultad:} Difícil\newline
\textbf{Descripción:} Reward for reaching the final of the spelling bee contest, you receive 500 in cash.\newline
\textbf{\textit{Trigger:}} \verb/R_T_M:500/\newline
\textbf{No es transferible por parte de los estudiantes}\newline
\textbf{No se debe imprimir código}
\end{tabular}
\tcblower
Da 500 puntos llegar a la final del concurso de deletreo que se hace en clase, su finalidad es motivar a los 
estudiantes a competir en este concurso, ya que es opcional su participación.
\end{tcolorbox}

\begin{tcolorbox}[colback=green!5!white,colframe=green!75!black,title=0008 - You delivered the work on time]
\begin{tabular}{ p{30mm} p{117mm}}
\adjincludegraphics[width=30mm,valign=t]{CALINA/simbolo_5}
&
\textbf{Tipo:} Recompensa\newline
\textbf{Dificultad:} Fácil\newline
\textbf{Descripción:} Reward for delivering work on time, you receive 50 in cash.\newline
\textbf{\textit{Trigger:}} \verb/R_T_M:50/\newline
\textbf{No es transferible por parte de los estudiantes}\newline
\textbf{No se debe imprimir código}
\end{tabular}
\tcblower
Da 50 puntos por entregar una actividad de la guía de aprendizaje a tiempo, su finalidad es estimular en los 
estudiantes la entrega puntual de una actividad.
\end{tcolorbox}

\begin{tcolorbox}[colback=green!5!white,colframe=green!75!black,title=0008 - You delivered a perfect job]
\begin{tabular}{ p{30mm} p{117mm}}
\adjincludegraphics[width=30mm,valign=t]{CALINA/simbolo_6}
&
\textbf{Tipo:} Recompensa\newline
\textbf{Dificultad:} Difícil\newline
\textbf{Descripción:} You have done a great job during the week. Congratulations!!!, you receive 500 in cash.\newline
\textbf{\textit{Trigger:}} \verb/R_T_M:500/\newline
\textbf{No es transferible por parte de los estudiantes}\newline
\textbf{No se debe imprimir código}
\end{tabular}
\tcblower
Da 500 puntos por entregar una actividad completa de la guía de aprendizaje, su finalidad es estimular en los 
estudiantes la entrega a satisfacción de una actividad.
\end{tcolorbox}

\uline{Cartas del catalogo de compra}, estas cartas siempre van a tener el estatus de compra al comienzo, 
por lo que se van a publicar desde el inicio de la guía para que el estudiante las adquiera, decida y planee 
cual desea comprar (gastar los puntos de experiencia). La importancia de su existencia, es otorgarle a los 
puntos de experiencia (dinero), un significado para el estudiante haciéndolos valiosos para los educandos, el 
criterio básico para seleccionar su costo esta dado por su dificultad e importancia que le da el docente, se 
debe seleccionar valores tal que el estudiante no pueda comprar todas las cartas de manera sencilla, y en lo 
particular que deba culminar todos los procesos de la guía para poder comprar las cartas mas valiosas. A 
continuación se publica las cartas creadas, su razón y costo.

\begin{tcolorbox}[colback=yellow!5!white,colframe=yellow!75!black,title=0009 - Improve exam grade]
\begin{tabular}{ p{30mm} p{117mm}}
\adjincludegraphics[width=30mm,valign=t]{CALINA/simbolo_7}
&
\textbf{Tipo:} Acción\newline
\textbf{Dificultad:} Normal\newline
\textbf{Descripción:} Increase the exam grade by one tenth.\newline
\textbf{\textit{Trigger:}} \verb/R_T_Q/\newline
\textbf{Costo:} 800\newline
\end{tabular}
\tcblower
El estudiante al usarla y validarla con el docente le sube la nota del examen de la guía en una décima. 
El educando puede comprar varias de estas cartas y usarlas para subir varias décimas la nota del examen\\
\end{tcolorbox}

\begin{tcolorbox}[colback=yellow!5!white,colframe=yellow!75!black,title=0010 - Remove the worst grade]
\begin{tabular}{ p{30mm} p{117mm}}
\adjincludegraphics[width=30mm,valign=t]{CALINA/simbolo_8}
&
\textbf{Tipo:} Acción\newline
\textbf{Dificultad:} Difícíl\newline
\textbf{Descripción:} Remove the worst grade of the term.\newline
\textbf{Costo:} 1500\newline
\end{tabular}
\tcblower
El estudiante al usarla y validarla con el docente le cambia la peor nota del trimestre por una nota promedio 
de las otras calificaciones.
\end{tcolorbox}

\begin{tcolorbox}[colback=yellow!5!white,colframe=yellow!75!black,title=0010 - I don't lose the exam]
\begin{tabular}{ p{30mm} p{117mm}}
\adjincludegraphics[width=30mm,valign=t]{CALINA/simbolo_9}
&
\textbf{Tipo:} Acción\newline
\textbf{Dificultad:} Difícíl\newline
\textbf{Descripción:} Ensure at least a grade of 3.0 in the final exam.\newline
\textbf{Costo:} 1800\newline
\end{tabular}
\tcblower
El estudiante al usarla y validarla con el docente le asegura mínimo un 3 en la nota definitiva.
\end{tcolorbox}

\begin{tcolorbox}[colback=yellow!5!white,colframe=yellow!75!black,title=0010 - I'm not gonna lose]
\begin{tabular}{ p{30mm} p{117mm}}
\adjincludegraphics[width=30mm,valign=t]{CALINA/simbolo_10}
&
\textbf{Tipo:} Acción\newline
\textbf{Dificultad:} Muy difícil\newline
\textbf{Descripción:} Ensure at least a grade of 3.0 in the term.\newline
\textbf{Costo:} 2500\newline
\end{tabular}
\tcblower
El estudiante al usarla y validarla con el docente le asegura mínimo un 3 en la nota definitiva.
\end{tcolorbox}

\begin{tcolorbox}[colback=green!5!white,colframe=green!75!black,title=0011 - I tried a lot]
\begin{tabular}{ p{30mm} p{117mm}}
\adjincludegraphics[width=30mm,valign=t]{CALINA/simbolo_11}
&
\textbf{Tipo:} Recompensa\newline
\textbf{Dificultad:} Muy difícil\newline
\textbf{Descripción:} Boost your ego! Congrats, you have put a lot of effort in the course!!!\newline
\textbf{No es transferible por parte de los estudiantes}\newline
\textbf{Costo:} 10000\newline
\end{tabular}
\tcblower
No sirve para ninguna aplicación práctica, pero es la carta mas costosa de la estrategia.
\end{tcolorbox}

\subsubsection{Capa 2. Diégesis y narración}

Esta capa busca la coherencia y cohesión entre los diferentes elementos de la estrategia implementada, por lo 
que el uso de una narrativa tiene como objeto aumentar el interés en la estrategia y por ende en la guía de 
aprendizaje, para está guía en particular se crea una historia sencilla para fortalecer la diégesis de los 
objetos.

Para desarrollar esto, se usan las mismas cartas de la capa 1, y se agregan algunos elementos adicionales como 
una imagen y una descripción que tiene narrativa, también se introducen cartas de tipo informativa que ayudan 
a contar algunas historias cortas para enriquecer el contexto de la estrategia. Las siguientes tarjetas son de 
uso exclusivo del grupo G\textsubscript{3}.

\begin{tcolorbox}[colback=blue!5!white,colframe=blue!75!black,title=00012 - Whispers from the past]
\begin{tabular}{ p{30mm} p{117mm}}
\adjincludegraphics[width=30mm,valign=t]{CALINA/openclipart/item129}
&
\textbf{Tipo:} Informativa\newline
\textbf{Dificultad:} Muy fácil\newline
\textbf{Descripción:} The past forgotten gods, wish to raise their voices again and be heard. Earlier, 
everyone lived in peace and harmony. Gods led to humans along the road  the way to wisdom.
\end{tabular}
\tcblower
Esta carta se introduce en esta capa para dar información adicional de la historia que se narra con las demás 
tarjetas
\end{tcolorbox}

\begin{tcolorbox}[colback=green!5!white,colframe=green!75!black,title=0001 - Attendance Card]
\begin{tabular}{ p{30mm} p{117mm}}
\adjincludegraphics[width=30mm,valign=t]{CALINA/openclipart/item230}
&
\textbf{Descripción:} The musical city has always had a beautiful pink color, and the imposing snow-capped 
Tolima has always accompanied us, just as you accompany us in this class. Welcome! As reward you have received 
200 Glorisetos.
\end{tabular}
\tcblower
Se selecciona una imagen que muestra un lugar donde viven los estudiantes, en este caso haciendo alegoría a la 
ciudad de Ibagué y se expande la descripción usando el tiempo en pasado.
\end{tcolorbox}

\begin{tcolorbox}[colback=blue!5!white,colframe=blue!75!black,title=0002 a 0005 - Trivia Card]
\begin{tabular}{ p{30mm} p{117mm}}
\adjincludegraphics[width=14mm,valign=t]{CALINA/openclipart/item109}
\adjincludegraphics[width=14mm,valign=t]{CALINA/openclipart/item108}\newline
\adjincludegraphics[width=14mm,valign=t]{CALINA/openclipart/item110}
\adjincludegraphics[width=14mm,valign=t]{CALINA/openclipart/item106}
&
\textbf{Descripción:} To use the power of the water/fire/air/earth element, you must communicate with the gods 
of the past, and speak their language, what is the past tense of  XXX?. Answer correctly and receive 50 
Glorisetos.
\end{tabular}
\tcblower
En general a todas las cartas se les pone una descripción que habla de su significado usando el tiempo pasado 
y se indica los puntos que hace al final.
\end{tcolorbox}

\begin{tcolorbox}[colback=green!5!white,colframe=green!75!black,title=0006 - Spelling Bee]
\begin{tabular}{ p{30mm} p{117mm}}
\adjincludegraphics[width=30mm,valign=t]{CALINA/openclipart/item189}
&
\textbf{Descripción:} A hard training have granted you a pass to speak with the gods. Well done! You have 
gotten 100 Glorisetos.
\end{tabular}
\end{tcolorbox}

\begin{tcolorbox}[colback=green!5!white,colframe=green!75!black,title=0007 - Spelling Bee finalist]
\begin{tabular}{ p{30mm} p{117mm}}
\adjincludegraphics[width=30mm,valign=t]{CALINA/openclipart/item249}
&
\textbf{Descripción:} Gods of the past have rewarded you eternal fortune for reaching the final of the 
spelling bee contest. They provide you with 500 Glorisetos.
\end{tabular}
\end{tcolorbox}

\begin{tcolorbox}[colback=green!5!white,colframe=green!75!black,title=0008 - You delivered the work on time]
\begin{tabular}{ p{30mm} p{117mm}}
\adjincludegraphics[width=30mm,valign=t]{CALINA/openclipart/item91}
&
\textbf{Descripción:} You are free! You have completed all your tasks on time. The ancient gods are grateful, 
whereby they give you 50 Glorisetos.
\end{tabular}
\end{tcolorbox}

\begin{tcolorbox}[colback=green!5!white,colframe=green!75!black,title=0008 - You delivered a perfect job]
\begin{tabular}{ p{30mm} p{117mm}}
\adjincludegraphics[width=30mm,valign=t]{CALINA/openclipart/item313}
&
\textbf{Descripción:} The ancient gods have given you the purple diamond, a rare gem found in the pink city. 
Having it gives you 500 Glorisetos.
\end{tabular}
\end{tcolorbox}

\begin{tcolorbox}[colback=yellow!5!white,colframe=yellow!75!black,title=0009 - Improve exam grade]
\begin{tabular}{ p{30mm} p{117mm}}
\adjincludegraphics[width=30mm,valign=t]{CALINA/openclipart/item166}
&
\textbf{Descripción:} The earth ancient god change your destiny. Now, It adds you a tenth to your final grade.
\end{tabular}
\end{tcolorbox}

\begin{tcolorbox}[colback=yellow!5!white,colframe=yellow!75!black,title=0010 - Remove the worst grade]
\begin{tabular}{ p{30mm} p{117mm}}
\adjincludegraphics[width=15mm,valign=t]{CALINA/openclipart/item239}
&
\textbf{Descripción:} The heavens god give you a bottle with one of his rays. So, go ahead! destroy your worst 
grade of the term.
\end{tabular}
\end{tcolorbox}

\begin{tcolorbox}[colback=yellow!5!white,colframe=yellow!75!black,title=0010 - I don't lose the exam]
\begin{tabular}{ p{30mm} p{117mm}}
\adjincludegraphics[width=30mm,valign=t]{CALINA/openclipart/item256}
&
\textbf{Descripción:} The waters ancient god has awarded you the knowledge to handle poisonous tests. Now, 
you are immune and you shall not fail in the final test.
\end{tabular}
\end{tcolorbox}

\begin{tcolorbox}[colback=yellow!5!white,colframe=yellow!75!black,title=0010 - I'm not gonna lose]
\begin{tabular}{ p{30mm} p{117mm}}
\adjincludegraphics[width=15mm,valign=t]{CALINA/openclipart/item316}
&
\textbf{Descripción:} The fire ancient god, gives you a rare feather originated by the flames of his crown, 
with  this one you won´t fail  the final grade of the term.
\end{tabular}
\end{tcolorbox}

\begin{tcolorbox}[colback=green!5!white,colframe=green!75!black,title=0011 - I tried a lot]
\begin{tabular}{ p{30mm} p{117mm}}
\adjincludegraphics[width=30mm,valign=t]{CALINA/openclipart/item16}
&
\textbf{Descripción:} Thanks for your effort! hours of hard work are now a thing of the past. Congrats, 
rise as a person blessed by the past gods.
\end{tabular}
\end{tcolorbox}

\subsubsection{Capa 3. Retos y extras}

Esta capa busca promover en los estudiantes la curiosidad y la motivación necesaria para ir mas allá del 
trabajo que propone la guía pedagógica, para lograr esto se hace uso de mecánicas como son los retos (para 
este caso se propone una carrera de observación) y el uso de huevos de pascua. Las siguientes tarjetas son de 
uso de ambos grupos G\textsubscript{2} y G\textsubscript{3}.

\begin{tcolorbox}[colback=blue!5!white,colframe=blue!75!black,title=00013 - Whispers from the past]
\begin{tabular}{ p{30mm} p{117mm}}
\adjincludegraphics[width=30mm,valign=t]{CALINA/simbolo_12}
&
\textbf{Tipo:} Informativa\newline
\textbf{Dificultad:} Muy fácil\newline
\textbf{Descripción:} The following clue is on a statue inside the school.
\end{tabular}
\tcblower
En una estatua dentro del colegio se pone un código QR con la carta número 14
\end{tcolorbox}

\begin{tcolorbox}[colback=blue!5!white,colframe=blue!75!black,title=00014 - Whispers from the past]
\begin{tabular}{ p{30mm} p{117mm}}
\adjincludegraphics[width=30mm,valign=t]{CALINA/simbolo_13}
&
\textbf{Tipo:} Informativa\newline
\textbf{Dificultad:} Muy fácil\newline
\textbf{Descripción:} Look for the secret password on the English teacher's desk and if you are the firt 
student to say this one to the whole class, you will receive a special and exclusive card.
\end{tabular}
\tcblower
La profesora te dará una tarjeta especial y única, si eres la primera persona que dice la contraseña
\end{tcolorbox}

\begin{tcolorbox}[colback=yellow!5!white,colframe=yellow!75!black,title=0015 - I'm a detective]
\begin{tabular}{ p{30mm} p{117mm}}
\adjincludegraphics[width=30mm,valign=t]{CALINA/simbolo_14}
&
\textbf{Tipo:} Acción\newline
\textbf{Dificultad:} Única\newline
\textbf{Descripción:} Add a unit to the final grade of the term.
\end{tabular}
\tcblower
El estudiante al usarla y validarla con el docente le agrega una unidad a la nota definitiva.
\end{tcolorbox}

\begin{tcolorbox}[colback=green!5!white,colframe=green!75!black,title=0016 - Easter egg]
\begin{tabular}{ p{30mm} p{117mm}}
\adjincludegraphics[width=30mm,valign=t]{CALINA/openclipart/item274}
&
\textbf{Tipo:} Recompensa\newline
\textbf{Dificultad:} Easy\newline
\textbf{Descripción:} The teacher’s other password is -rubber duck-\newline
\textbf{No es transferible por parte de los estudiantes}\newline
\textbf{Costo:} 400\newline
\end{tabular}
\tcblower
Esta carta se encuentra escondida en el acuerdo pedagógico subido a la plataforma institucional, y trae una 
contraseña para recibir una carta especial 0017
\end{tcolorbox}

\begin{tcolorbox}[colback=yellow!5!white,colframe=yellow!75!black,title=0017 - Help me]
\begin{tabular}{ p{30mm} p{117mm}}
\adjincludegraphics[width=30mm,valign=t]{CALINA/openclipart/item258}
&
\textbf{Tipo:} Acción\newline
\textbf{Dificultad:} Easy\newline
\textbf{Descripción:} The English teacher gives you an extra help in your final test.
\end{tabular}
\tcblower
El estudiante al usarla y validarla con el docente le regala una ayuda extra en su examen
\end{tcolorbox}
