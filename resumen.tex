\section*{Resumen}

\begin{singlespace}
El siguiente proyecto surge de la necesidad de brindar una estrategia didáctica mediada por una aplicación 
móvil para el seguimiento y motivación de las actividades en el aula \cite{SAILER2017371, DAROCHASEIXAS201648}, 
con el objetivo de promover el desarrollo de buenos hábitos de estudio y aumentar el éxito en la terminación 
de actividades por parte de los estudiantes, bajo este contexto, el presente proyecto probará un enfoque 
interactivo del proceso de aprendizaje diseñando y probando una herramienta que involucra elementos de 
gamificación y narración de historias en forma de juegos de rol \cite{rauscher2021comics}.

Para el desarrollo del presente trabajo se construye una tecnología de software de nombre CALINA (en idioma 
Pijao significa Amigo, disponible para su descarga en 
\url{https://play.google.com/store/apps/details?id=co.edu.ut.jrbustosm.calina}), que media el progreso de las 
actividades estudiantiles haciendo uso de técnicas de gamificación y juegos de rol, con el fin de promover la 
apropiación de competencias en cuanto a aprender a aprender y lograr así un desarrollo integral de quien 
consolida un proceso mediado por la aplicación \cite{tornero2016ideas, molina_reconfiguracion_2021}, 
fomentando en los estudiantes la responsabilidad en la ejecución de actividades y el trabajo en equipo 
\cite{XU2017}.

El uso extensivo de técnicas de gamificación permite que el estudiante se encuentre motivado a explorar y 
continuar las actividades propuestas encontrando satisfacción en sus propias conquistas y logros 
\cite{Danka2020, MULLINS2020304}, y al poder ayudar a sus compañeros para que alcancen los mismos objetivos 
que tienen en común promueve valores de unión y solidaridad en el grupo \cite{DING20191}, adicionalmente, el 
planteamiento de actividades bajo un esquema de narración de historias y juegos de rol enriquece contextos 
aburridos del aula estimulando, inspirando y motivando a los estudiantes, especialmente si la historia está en 
concordancia con sus intereses personales lo que interioriza las competencias propuestas 
\cite{8190501, Young2015199}. 

En cuanto al desarrollo metodológico del presente proyecto, presenta dos etapas, una primera del desarrollo de 
la aplicación, implementada inicialmente usando un método de cascada seguido por uno en base a prototipos para 
la construcción del software, y el modelo MAP (Mecánicas, Atributos y Principios) para la adquisición de 
requisitos en cuanto a gamificación \cite{CECHELLA2018}, y una segunda etapa de verificación de tipo 
cualitativo que implementa un diseño cuasi experimental, en el que se van a estudiar la influencia de la 
``gamificación'' y ``juegos de rol'' en los estudiantes desde la perspectiva del docente, para esto, se aplica 
las diferentes estrategias sobre tres grupos diferentes manejados por un único profesor, cada grupo con su 
propia estrategia, (1) grupo de control (2) grupo con gamificación y (3) grupo con gamificación + juego de rol, 
cada uno con dos replicas que consta de sesiones cortas de 30 a 45 minutos. 

La pregunta de investigación radica en que tanto la gamificación como los juegos de rol afectan las emociones 
de los estudiantes \cite{MULLINS2020304}, afectando su motivación y asimilación de las competencias 
propuestas, por lo que se plantean dos instrumentos para su verificación, (1) La aplicación CALINA en si como 
instrumento de medición para hacer seguimiento de las actividades y (2) entrevista no estructurada
al inicio y final de la investigación al docente de la asignatura. 

Finalmente, lo que se pretende con esta investigación es hacer reflexión de como cambiar esquemas de 
aprendizaje, puede brindar a los estudiantes motivaciones intrínsecas (deseo hacerlo) en lugar de las 
motivaciones extrínsecas que suelen darse al estudiante (lo hago por la nota), esto con el fin de generar los 
hábitos de trabajo necesarios en esforzarse para culminar y continuar con algo que deseo hacer, en lugar del 
típico esquema de esforzarse con el fin de superar algo que el profesor desea que haga.
\end{singlespace}

\uline{Palabras clave:} gamificación, narración de historias, story telling, aplicación móvil, juegos de Rol, RPG

%\section*{Abstract}

%\begin{singlespace}
%\end{singlespace}

%\uline{Keywords:} gamification, story telling, mobile application, Role Playing, RPG

