% Autor: Jose Ricardo Bustos Molina
%        Universidad del Tolima
%        jrbustosm@ut.edu.co
%
\section{Metodología}

De acuerdo al modelo MAP (Ver Figura \ref{img:MAP}) seleccionado para el desarrollo de este trabajo y las 
revisión de antecedentes, se formulan los siguientes principios para su desarrollo metodológico del presente
trabajo, con el fin de guiar el diseño y desarrollo en la aplicación de las estrategias planteadas,

\begin{table}[ht]
\caption{Principios para el desarrollo de la estrategia gamificada}
\label{tab:principios}
\begin{center}
	\begin{tabular}{ p{150mm}}
\toprule
\textbf{Principios} \\ 
\midrule
\textbf{P0:} El Jugador es el centro de todo, permitir alcanzar sus propios objetivos, por lo tanto, adaptar 
los elementos del juego de acuerdo con su perfil es una forma de mejorar su experiencia al interactuar con el 
sistema gamificado \cite{Klock2020}\\
\textbf{P1:} La gamificación no establece el tema de clase, se debe ajustar a su contenido y brindar un 
contexto diferente\\
\textbf{P2:} Los estudiantes tienen control sobre el juego no son pasivos, sus acciones importan para su
desarrollo y expansión\\
\textbf{P3:} Los estudiantes tienen la autonomía de ganar o perder, perder debe hacer parte importante del 
juego\\
\textbf{P4:} Es importante hacer que jueguen (aprendan), pero el verdadero objetivo es que sigan jugando
(aprendiendo)\\
\textbf{P5:} Se debe crear experiencias en lugar de solo dar puntos, debe tener un fin que justifique 
jugarlo, por lo que esos puntos deben significar algo para el estudiante\\
\textbf{P6:} El juego debe ser VOLUNTARIO no obligado, a nadie le gusta que lo obliguen a jugar\\
\textbf{P7:} El juego debe ser desafiante mas no imposible, nadie desea jugar sin el nivel de reto adecuado\\
\textbf{P8:} El juego debe indicar cuando se están haciendo las cosas bien y las cosas mal, en el debido 
momento, retroalimentación\\
\textbf{P9:} Se debe realizar siempre un nivel de introducción o de aprendizaje del juego, para conocer sus 
mecánicas y reglas, estas deben ser lo suficientemente claras y sencillas para poder jugarlo\\
\textbf{P10:} La gamificación debe conectar a los jugadores para que se apoyen entre sí y trabajen hacia un 
objetivo común, creando su propia identidad en el grupo.\\
\bottomrule
{\footnotesize Fuente: de elaboración propia.}
\end{tabular}
\end{center}
\end{table}

\subsection{Método o estrategias metodológicas}	

\begin{description}
\item[Tipo: Investigación a través de la educación] \hfill \\ El tipo de investigación es un 
desarrollo tecnológico cuyo producto final es un software, ya que busca mediar en un proceso pedagógico 
práctico para la culminación efectiva de las labores escolares, utilizando métodos gamificación y juegos de 
rol con el fin de motivar a los estudiantes.
\item[Diseño: Investigación experimental] \hfill \\ Se usa un diseño experimental ya que el objetivo 
de la investigación parte de observar los efectos que se tienen en los estudiantes las técnicas de 
gamificación y juegos de rol.
\item[Enfoque: Cualitativo] \hfill \\ El enfoque de la investigación cualitativa se usa para entender 
el comportamiento de los estudiantes en dos aspectos: culminación de actividades y su motivación para hacerlas 
desde la perspectiva del docente, por lo que se implementan dos encuestas no estructuradas con el docente con 
el fin de entender los motivos, significados y las razones internas del comportamiento del estudiante hacia la 
actividad y así adquirir una comprensión de la influencia de la gamificación y juegos de rol en ellos.
\end{description}

\subsection{Unidad de análisis}

La unidad de análisis para esta investigación van a ser personas, en este caso un único docente. Como se va a
desarrollar desde una perspectiva cualitativa el tamaño de muestra realmente no es importante, debido a que en 
este tipo de investigación el objetivo no es hacer una generalización de los resultados del estudio o 
establecer una teoría, si no explicar un suceso que ocurre en el aula. Por lo tanto el tamaño lo determina el 
nivel de acceso que se tiene sobre los sujetos de estudio.

Para esta investigación el docente va a trabajar con tres cursos completos de estudiantes de grado noveno de 
un colegio urbano público, de una única institución escolar del municipio de Ibagué (Tolima). Debido a que los 
sujetos de investigación no son los estudiantes, y la gamificación no es obligatoria, esta estrategia se  
aplica solamente con los estudiantes que deseen y tengan la posibilidad de ejecutarla, . La Tabla 
\ref{tab:cuasi} muestra la disposición de los tres cursos.

\begin{table}[h]
\caption{Disponibilidad de cursos por parte del docente}
\label{tab:cuasi}
\begin{center}
\begin{tabular}{ p{30mm} p{100mm} }
\toprule
	\textbf{Grupo} & \textbf{Tratamiento} \\
\midrule
	\textbf{G\textsubscript{1}} & \textbf{(-)} Grupo de con metodología normal\\
	\textbf{G\textsubscript{2}} & \textbf{(X\textsubscript{1})} Estrategia Gamificada\\
	\textbf{G\textsubscript{3}} & \textbf{(X\textsubscript{2})} Estrategia Gamificada + Juego de Roles\\
\bottomrule
	\multicolumn{2}{l}{\footnotesize Fuente: de elaboración propia.}\\
\end{tabular}
\end{center}
\end{table}

\subsection{Técnicas e instrumentos de investigación}

\begin{description}
\item[Software CALINA] \hfill \\ El desarrollo del vídeo juego en si, es un instrumento para 
recopilar información explicita de los estudiantes a través de sus acciones en el juego. Se plantea poder 
medir, tiempos, número de accesos, ruta de aprendizaje seleccionada, progreso y logros.
\item[Entrevista] \hfill \\ Se propone una entrevista no estructurada al docente en el inicio de la 
investigación para permitir identificar y documentar los requerimientos de la estrategia a implementar, y 
ajustar el desarrollo de la aplicación, con base a las intenciones del docente con la temática a tratar. 
Adicional, se propone una entrevista no estructurada al final para recolectar información de las impresiones
del docente respecto al cambio en sus estudiantes.
\end{description}

\subsection{Técnicas de análisis de datos}

\begin{description}
\item[Estadística descriptiva] \hfill \\ Se utiliza una técnica descriptiva sobre las entrevistas no 
estructuradas.
\end{description}

\subsection{Alcances y limitaciones de la presente metodología}

\begin{description}
\item[Alcances] \hfill \\ El software no tiene la capacidad de almacenar las acciones realizadas por los 
estudiantes, ni de enviar ningún registro sobre su uso a un nodo central.

De acuerdo al principio \textbf{P1} (ver Tabla \ref{tab:principios}) el presente trabajo no puede imponer un
contenido a la clase adicional al propuesto por el docente del aula, ya que no es el fin de las estrategias
gamificadas o de juegos de rol.

\item[Limitaciones] \hfill \\ Ya que depende del uso de la aplicación por parte de los estudiantes para 
recopilar información se ve limitado por esta, al momento del desarrollo no es posible saber cuales son las 
intenciones por las cuales se usa y los sentimientos que esto genera, por lo que es necesario contrastar con 
otras técnicas como la observación y las encuestas. Adicional se requiere que el estudiante o el colegio 
cuente con el hardware para implementar las estrategias que así lo requieran.

De acuerdo al principio \textbf{P6} (ver Tabla \ref{tab:principios}) auto impuesto para el desarrollo del
presente trabajo, solo pueden participar los estudiantes que voluntariamente deseen implementar el juego, ya
que una medida de obligatoriedad impondría ruido al establecer una motivación extrínseca como base, por lo que
es posible que no todos los miembros de un grupo participen.
\end{description}

\subsection{Proceso metodológico propuesto}

El proceso metodológico propuesto para el presente trabajo se puede ver en la Figura \ref{img:metodologia} 
toma elementos del Modelo MAP (ver Figura \ref{img:MAP}), la metodología de cascada y por prototipos para el 
desarrollo del software (aplicación para móviles) y el proceso de investigación cualitativo tradicional. En 
síntesis, se inicia con una introducción al ambiente, tomando nota del contexto donde y quienes van a aplicar 
las pruebas, obteniendo los permisos o consentimientos correspondientes, para luego proceder con una 
entrevista de tipo no estructurada al docente con el fin de obtener requisitos para la mejora en el desarrollo 
del software y las estrategias a emplear en los cursos sujetos a esta investigación.

Posterior a la construcción del software, se toma la guía de aprendizaje realizada por el docente del área,
ya que es la aplicación la que debe ajustarse a los temas que normalmente se dan (ver principio \textbf{P1} 
mostrado en la Tabla \ref{tab:principios}). Para esta investigación, se van a aplicar tres estrategias, una de
control, una segunda gamificada y otra gamificada usando elementos de los juegos de rol, cada estrategia se
aplica a un grupo diferente en dos sesiones (replicas), bajo la salvedad que los dos grupos con estrategia 
gamificadas se realiza previamente una misión corta de entrenamiento (ver principio \textbf{P9} mostrado en la 
Tabla \ref{tab:principios}).

En conclusión, en la Figura \ref{img:metodologia} se puede inferir que se sigue un \textbf{diseño 
fenomenológico}, ya que el estudio busca comprender desde la perspectiva de un docente las experiencias y 
emociones de un grupo de estudiantes, los cuales se ven expuestos a una estrategia gamificada o con juegos de 
rol. Por lo que al final se desea conseguir una descripción sobre como la aplicación de este tipo de 
estrategias afectan la experiencia docente y e influye en los sentimientos de varios de los participantes del 
estudio, identificando las características que se tienen en común.

\begin{figure}[htp]
\caption{Modelo del proceso metodológico propuesto}
\label{img:metodologia}
\centering
\begin{tikzpicture}[
		every node/.append style={draw=gray!80,align=center,minimum width=90pt,very thin}
  	]
	\node (I) at (0,13.5) {
		\textbf{\small Inicio}
	};
	\node (IA) at (0,12) {
		\textbf{\small Introducción al ambiente}\\
		{\scriptsize Contexto, participantes y control}
	};
	\node (C) at (0,10) {
		\textbf{\small Consentimiento del docente}\\
		{\scriptsize Participantes y colegio}
	};
	\node (E) at (0,8) {
		\textbf{\small Entrevista no estructurada}\\
		{\scriptsize Requisitos del software y de la estrategía}
	};
	\node (DG) at (0,6) {
		\textbf{\small Diseño guía aprendizaje}\\
		{\scriptsize Construida por docente}
	};
	\node (DC) at (6,10) {
		\textbf{\small Desarrollo del Software}\\
		{\scriptsize Método de la cascada}
	};
	\node (D) at (6,8) {
		\textbf{\small Desarrollo del Software}\\
		{\scriptsize Método por prototipo}
	};
	\node (SG) at (6,6) {
		\textbf{\small Diseño de mecánicas}\\
		\textbf{\small y atributos}\\
		{\scriptsize Gamificación}
	};
	\node (SR) at (12,6) {
		\textbf{\small Diseño mythos,}\\
		\textbf{\small ethos y topo}\\
		{\scriptsize Juego de rol}
	};
	\node (OG) at (6,4) {
		\textbf{\small Misión de entrenamiento}\\
		{\scriptsize G\textsubscript{2}}
	};
	\node (OR) at (12,4) {
		\textbf{\small Misión de entrenamiento}\\
		{\scriptsize G\textsubscript{3}}
	};
	\node (S1C) at (0,2) {
		\textbf{\small Aplicación}\\
		{\scriptsize G\textsubscript{1} - O\textsubscript{1}}
	};
	\node (S1G) at (6,2) {
		\textbf{\small Aplicación}\\
		{\scriptsize G\textsubscript{2} X\textsubscript{1} O\textsubscript{2}}
	};
	\node (S1R) at (12,2) {
		\textbf{\small Aplicación}\\
		{\scriptsize G\textsubscript{3} X\textsubscript{2} O\textsubscript{3}}
	};
	\node (EG) at (6,0) {
		\textbf{\small Entrevista no estructurada}\\
		{\scriptsize Impresiones sobre estudiantes}
	};
	\node (A) at (6,-2) {
		\textbf{\small Análisis de Datos}
	};
	\draw[-triangle 90] (I) edge (IA);
	\draw[-triangle 90] (IA) edge (C);
	\draw[-triangle 90] (C) edge (E);
	\draw[-triangle 90] (DC) edge (D);
	\draw[-triangle 90,dashed] (DC) edge (E);
	\draw[-triangle 90] (E) edge (D);
	\draw[-triangle 90] (E) edge (DG);
	\draw[-triangle 90] (D) edge (SG);
	\draw[-triangle 90] (D) edge (SR);
	\draw[-triangle 90,dashed] (DG) edge (SG);
	\draw[-triangle 90,dashed] (DG) -- ++(3cm,0cm) -- ++(0,1.2cm) -| (SR);
	\draw[-triangle 90] (SG) edge (OG);
	\draw[-triangle 90] (SR) edge (OR);
	\draw[-triangle 90] (DG) edge (S1C);
	\draw[-triangle 90] (S1C) edge (EG);
	\draw[-triangle 90] (OG) edge (S1G);
	\draw[-triangle 90] (S1G) edge (EG);
	\draw[-triangle 90] (OR) edge (S1R);
	\draw[-triangle 90] (S1R) edge (EG);
	\draw[-triangle 90] (EG) edge (A);
\end{tikzpicture}
\\
{\footnotesize Fuente: de elaboración propia}
\end{figure}
