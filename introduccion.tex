% Autor: Jose Ricardo Bustos Molina
%        Universidad del Tolima
%        jrbustosm@ut.edu.co
%
\section{Introducción}

\subsection{Planteamiento del problema de investigación}	

En el estudio ``Factores asociados al desempeño académico en la  prueba saber 3°, 5° y 9° - 2012'' realizado 
por \citeA{ICFES_2018}, el gobierno nacional presenta la necesidad de mitigar o mejorar en diversos factores 
que influencian directamente el rendimiento escolar, entre varios mencionados las estrategias de aprendizaje y 
el autoconcepto académico, son dos factores que están estrechamente relacionados con disminuir la falta de 
motivación en los estudiantes.

En el día a día del ejercicio docente en Colombia es un hecho, que numerosos estudiantes de secundaria se 
encuentran en un estado donde presentan un bajo deseo en realizar sus tareas académicas, es esta ausencia de 
motivación académica que puede generar sentimientos de frustración y descontento, afectando su productividad y 
el bienestar de ellos mismos \cite{Legault2006-af}. Aunque la motivación académica siempre ha recibido mucha 
atención por parte de los propios docentes, no se ha prestado suficiente atención a las razones por la falta 
de esta, ya que las causas de su carencia es en sí mismo un proceso muy complejo, que no es tanto una ausencia 
de motivación externa hacia el estudiante, si no que incluye un amplio campo de necesidades internas no 
satisfechas en los mismos estudiantes y su entorno.

Es esta naturaleza multidimensional de la desmotivación académica la que explica el comportamiento de los 
estudiantes, y es descrita por \citeA{deci1985intrinsic}, quienes con su modelo de la teoría de 
autodeterminación listan cuatro razones: sus creencias sobre su propia capacidad, las creencias sobre el 
esfuerzo, el valor que se le da a las tareas académicas y las características de las tareas académicas en sí. 

Por lo que los estudiantes que experimentan una situación donde sus capacidades o esfuerzo estén disminuidas
creerán a priori que su situación es permanente y que no hay nada que puedan hacer al respecto, asumiendo 
erróneamente que son los factores externos quienes controlan su destino, experimentando una pérdida de control 
y una sensación general de impotencia. Es por esto, que los esfuerzos docentes para motivar deben ir 
encaminados a aumentar en los estudiantes esta confianza en sus propias habilidades y el esfuerzo, para tener 
éxito en la culminación de su propio currículo.

Por otro lado, el valor otorgado a las actividades académicas esta directamente relacionado con los deseos de
los estudiantes de abandonar o continuar con ellas, es por esto que es necesario que los estudiantes y sus 
seres queridos (familia), valoren abiertamente el éxito académico. Los estudiantes necesitan identificarse con 
estas tareas las cuales requieren tiempo y esfuerzo. En otras palabras, Si los estudiantes valoran lo que 
están haciendo, es probable que se comprometan.

Finalmente las tareas escolares deben ser inspiradoras e interesantes, y los estudiantes deben sentirse 
identificados para realizarlas. Es por esto, que la percepción de los estudiantes a actividades poco 
interesantes, aburridas, o monótonas deben ser reexaminadas en un intento de hacerlas más atractivas. 
Adicionalmente, la importancia que los docentes brinden a sus estudiantes la información y la 
retroalimentación necesaria y oportuna, para impulsar la motivación académica en sus estudiantes.

\subsection{Justificación del problema de investigación}

Hoy en día, uno de los mayores retos para los docentes es lograr construir entornos pedagógicos efectivos y 
atractivos para los estudiantes. Según \citeA{AURA2021101728} gracias a la digitalización de la vida cotidiana 
y una creciente inmigración tanto física como digital, los grupos de estudiantes no solo son cada vez más 
heterogéneos (en términos de raza, género y religiones), sino que sus intereses tanto individuales como de 
colectivo hacen que su capacidad de atención sea disminuida significativamente ya que lo visto en su escuela 
no se ajusta a sus gustos.

Es así como en los últimos años con el aumento en el uso por parte de la comunidad educativa de las redes 
sociales, las plataformas de vídeos y la difusión generalizada de juegos, se ha vuelto casi indispensable para 
los docentes utilizar algunas de las mismas tecnologías que le interesan a los estudiantes para con ellas 
crear entornos de aprendizaje interesantes, atractivos y familiares. Por lo que realizar acciones para mejorar 
el intereses en el aprendizaje e inspirar una enseñanza más apropiada juega un papel importante para los
modelos de aprendizajes modernos \cite{XU2017}. Creando la necesidad de realizar un aprendizaje activo como 
una actividad cotidiana en el aula, lo que requiere que los estudiantes hagan algo más que escuchar y tomar 
notas \cite{8190501}.

%GAMIFICACION
Según \citeA{KUSUMA2021886} al aumentar la motivación también es posible mejorar el rendimiento escolar. Para 
poder lograr este incremento en la motivación escolar es posible aplicar estrategias que usen gamificación, 
concretamente, el desarrollo de prácticas para involucrar y estimular a los estudiantes en el proceso de 
aprendizaje, así como para mejorar su experiencia en el aula \cite{SBIE8805}.

De igual manera \citeA{duran2019}, definen que ``a través del juego se enfrenta el individuo a diferentes 
desafíos y experiencias que tienen que superar aprendiendo de sus experiencias'' es así como la gamificación 
brinda posibilidades en la automotivación, la inquietud por el saber, y la relevancia del juego para 
aplicarlos en la educación, logrando así, un mayor grado de participación en el estudiante y opciones de 
mejora en su rendimiento escolar. Por lo que, el juego se convierte en uno de los medios más poderosos que 
tienen los niños para aprender nuevas habilidades y conceptos a través de su propia experiencia.

Se han realizado estudios sobre el uso de gamificación en la enseñanza y el aprendizaje, y se ha observado la 
influencia positiva de este recurso en áreas como la motivación, el compromiso y el aprendizaje de los 
estudiantes \cite{DING20191}. Sin embargo, las estrategias basadas en gamificación no siempre dan como 
resultado experiencias positivas, existiendo estudios donde no hay ningún efecto o incluso reportando efectos 
negativos, por lo que, un diseño meticuloso contribuye en gran medida al éxito de un proceso que usa este tipo
de técnicas \cite{DING20191}.

%NARRACION
Por otro lado, para este trabajo es indispensable considerar la narrativa como una herramienta que afecta
directamente a los sentidos y emociones de los estudiantes, favoreciendo o afectando el aprendizaje, al 
ofrecer mediante historias la posibilidad de contextualizar las enseñanzas. La narrativa permite unir ideas 
sueltas dentro de las actividades en clase, lo que facilita el recuerdo, la asociación y la transferencia de 
conocimiento \cite{tornero2016ideas}.

Sin embargo, según \citeA{Young2015199} la investigación sobre el papel de la narrativa en los juegos y en el 
aprendizaje en el aula está lejos de ser concluyente, pero son un determinante central del aprendizaje 
humano, ya que la narrativa es el mecanismo a través del cual los humanos construyen la realidad y dan sentido 
al mundo que los rodea. Es por esto que el uso de narrativas y cómo estas pueden apoyar la implementación de 
la gamificación fomentando y desarrollando la creatividad, el interés y apropiación en el aula, es un campo
de estudio que debe ser estudiado y evaluado.

%RPG
Finalmente, el uso propuesto de elementos de juegos de rol (RPG) clásicos es con el fin de proporcionar una 
estrategia integrada y de múltiples frentes para involucrar a los estudiantes, que sirva de elemento de 
cohesión entre la gamificación, narrativa y competencia a aprender, logrando transmitir la información de
una forma interesante para alentarlos a crecer y mejorar.

\subsection{Objetivos de la investigación}

\subsubsection{Objetivo general}

Construir y evaluar una aplicación gamificada para móviles que usa juegos de rol para el desarrollo de 
actividades escolares

\subsubsection{Objetivos específicos}

\begin{itemize}
\item Identificar las necesidades que presentan los estudiantes, para el buen desarrollo o culminación de sus
actividades escolares.

\item Seleccionar las estrategias adecuadas de gamificación y de juegos de rol adecuadas de acuerdo al 
contexto para su aplicación en un proceso educativo.

\item Evaluar mediante instrumentos de investigación para dimensionar la motivación el modo en que las 
técnicas de gamificación y juegos de rol afectan el comportamiento de los estudiantes.
\end{itemize}

\subsection{Pregunta de investigación}

¿En que medida afecta el uso de estrategias gamificadas y basadas en juegos de rol a la motivación y la
culminación de actividades en los estudiantes?

\subsection{Categorías de investigación}

\begin{table}[ht]
\caption{Categorías de Investigación}
\label{tab:catinv}
\small
\begin{center}
	\begin{tabular}{ p{40mm} p{30mm} p{30mm} p{40mm}}
\toprule
\textbf{Pregunta} & \textbf{Categorías} & \textbf{Subcategorías} & \textbf{Técnicas} \\ 
\midrule
¿Cómo afecta la gamificación en la motivación y culminación de actividades? & Gamificación & Mecánicas 
	\newline Dinámicas \newline Diégesis & Aplicación CALINA \newline Implementación de estrategia 
	gamificada\\
¿Cómo afecta los juegos de rol en la motivación y culminación de actividades? & Narración de Historias & 
	Juegos de Rol & Aplicación CALINA \newline Implementación de juego de rol\\
¿Cómo medir el grado de motivación en un estudiante? & Teorías Motivación & Autodeterminación \newline 
	Culminación \newline Cooperativo \newline Retroalimentación \newline Individualización & Aplicación 
	CALINA \newline Observación participante ajena\newline Diseño encuesta\\
\bottomrule
\multicolumn{4}{l}{\footnotesize Fuente: de elaboración propia.}\\
\end{tabular}
\end{center}
\end{table}

