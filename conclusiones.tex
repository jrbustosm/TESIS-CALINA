% Autor: Jose Ricardo Bustos Molina
%        Universidad del Tolima
%        jrbustosm@ut.edu.co
%

\section{Conclusiones}

\textbf{Revisión de antecedentes}

Se desarrollo para este proyecto un proceso sistemático para la búsqueda de antecedentes, en la que las 
cadenas de búsqueda se mutaban de acuerdo a los artículos seleccionados mediante estrategias como inteligencia 
artificial y el uso de nubes de palabras o conceptos, encontrando campos que no se habían considerado a 
priori, lo que enriqueció la base de datos de artículos seleccionados para el presente trabajo;a parte se hizo 
una búsqueda exhaustiva en las bases de datos de las universidades Colombianas creando una bibliografía 
anotada sobre el tema de gamificación en Colombia a nivel de maestría y doctorado.

\textbf{Metodología}

Es de exaltar la creación de una tabla de principios para el desarrollo e implementación de la estrategia 
gamificada en el presente documento, en la cual se exponen 11 principios básicos, estos principios son la base 
sobre la cual se construye el software, la metodología de investigación y la planificación pedagógica.

Por otro lado se hizo un diseño para implementar un ensayo en la aplicación del software CALINA, así como la 
conceptualización de los instrumentos necesarios a usar, para este primer ensayo de la aplicación construida 
durante el desarrollo del trabajo.

\textbf{Desarrollo de software}

Se desarrollo un software para la plataforma Android, el cual permite implementar el uso de gamificación en el 
desarrollo normal de una clase, usando mecánicas como puntos de experiencia, recompensas, retos, huevos de 
pascua, presión de tiempo, juego cooperativo, trofeos, bienes virtuales, intercambio, lotería, exaltaciones,
comercio, equipos, misión de entrenamiento y elementos desbloqueables. El cual permite a los estudiantes 
obtener beneficios por sus acciones y tener el control sobre lo que puede pasar en su entorno, todo esto bajo 
la restricción de no hacer uso de los datos, por lo que todo, la transmisión de información se hace por medio 
de la cámara y códigos QR.

\textbf{Propuesta pedagógica}

Se realizó una discusión sobre como se debe implementar una estrategia gamificada al currículo docente, usando 
los modelos existentes para el desarrollo curricular, y se crea e implementa un modelo por capas para lograr
que una estrategia gamificada se pueda aplicar a una guía sin tener que modificar a esta, y por el contrario
la complemente y enriquezca. Seguido a esto, se implemento un conjunto de tarjetas de CALINA para motivar y 
recompensar a los estudiantes del curso de inglés del grado noveno del colegio público Ciudad Luz.

\textbf{Resultados implementación}

La investigación llevada a cabo, tuvo como objetivo desarrollar una estrategia gamificada mediada por una 
aplicación móvil creada en el mismo desarrollo de la investigación de nombre CALINA, con el fin de motivar 
e inspirar a los estudiantes, para ello se seleccionó a los grados noveno de la Institución pública Educativa 
Ciudad Luz en Ibagué, la indagación estuvo dirigida al nivel de la perspectiva de la docente del curso de 
Inglés, dado que se pretendía hacer una primera aproximación del entendimiento del uso de esta estrategia y 
como ella influye en los estudiantes desde la perspectiva del profesor, logrando evidenciar un aumento en la
motivación, rendimiento académico y la participación de los educandos gracias al uso de la estrategia.

\textbf{Algunas sugerencias para investigaciones a futuro}

Es recomendable expandir y formalizar los desarrollos del modelo de capas para el desarrollo de guias de 
aprendizaje o currículos gamificados y los 11 principios, los cuales fueron compilados o desarrollados por el 
autor del presente trabajo.

Se recomienda ampliar la investigación cualitativa donde se incluyan más docentes e incluso que el objeto de
estudio sean los estudiantes, y se apliquen mas instrumentos de investigación para medir la motivación, 
sentimientos y demás características que influyen en un buen proceso de aprendizaje.

Existe la necesidad de expandir la opción de ``triggers'' en la aplicación, así como hacer un formulario que 
facilite su incorporación a las cartas creadas, también se recomienda incluir la mecánica de los nivel de 
reputación para una versión posterior de la aplicación.

Es necesario para la publicación del software a la comunidad docente y académica, que este sea publicado bajo 
una licencia de software libre tipo GNU, para aumentar su aceptación y visibilidad.

Por último, a pesar de que CALINA fue concebido desde el inicio como una aplicación que no usa los datos, se 
puede desarrollar una versión CALINA-WEB que si lo haga, expandiendo enorme mente las capacidades del concepto 
de CALINA, permitiendo una transmisión de cartas mas rápida y global, el uso de nuevas mecánicas, como tablas 
de clasificación, árbol de habilidades, avatares, records, barras de progreso entre otras; también, el uso 
de este medio de transmisión aumenta la capacidad de CALINA ya que no estaría limitado a los 4.000 caracteres 
que si tienen los códigos QR.

