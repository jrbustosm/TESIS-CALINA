% Template:     Tesis LaTeX UT
% Versión:      1.0 (12/01/2022)
% Codificación: UTF-8
%
% Autor: Jose Ricardo Bustos Molina.
%        Maestría en pedagogía y mediaciones tecnológicas
%        Universidad del Tolima
%        jrbustosm@ut.edu.co

% Codificación
\usepackage[utf8]{inputenc}
\usepackage[T1]{fontenc} % Caracteres acentuados
\inputencoding{utf8}

% Carga el idioma
\usepackage[spanish,es-nosectiondot,es-lcroman,es-noquoting]{babel}

% -----------------------------------------------------------------------------
% Paquetes para dependencias
% -----------------------------------------------------------------------------
\usepackage{float}         % Administrador de posiciones de objetos, permite poner imágenes donde son creadas
\usepackage{datetime}      % Fechas
\usepackage{lipsum}        % Permite crear párrafos de prueba
\usepackage{rotating}      % Permite rotación de objetos
\usepackage{url}           % Permite añadir enlaces
\usepackage{color}         % Colores
\usepackage{amsmath}       % Manejo de ecuaciones matemáticas
\usepackage{verbatim}      % Para comentar y mostrar también código
\usepackage{afterpage}     % Para ejecutar comandos en la siguiente pagina
\usepackage{ifthen}
\usepackage{xcolor}       % utilidades para manejo de colores
\usepackage{stackengine}  % Para apilar objetos, y ubicarlos

% -----------------------------------------------------------------------------
% Configuración de margenes y tamaño de página
% -----------------------------------------------------------------------------
\usepackage{adjustbox}     % Para hacer cajas para ajustar los objetos y re-dimensionarlos
\usepackage{varwidth}      % Similar a minipage 
\usepackage{geometry}
\geometry{letterpaper, margin=1in, portrait} %includehead, includefoot

% -----------------------------------------------------------------------------
% Configuración de tamaño y estilo de la fuente del documento
% -----------------------------------------------------------------------------
\usepackage{scrextend}       % Paquete complejo de scripts koma, lo uso solo para cambiar tamaño de fuente
\usepackage{setspace}        % Cambia el espacio entre líneas
\usepackage[normalem]{ulem}  % Permite tachar y subrayar
\usepackage{csquotes}        % Citas y comillas, se debe usar después de lineno [6.4.2]
\usepackage{fontawesome}     % iconos nuevos 
\usepackage{textcomp}        % Algunos símbolos de común uso
\usepackage{ccicons}         % iconos de licencias creative commons
\usepackage{helvet}          % Para Fuente Arial

\renewcommand{\familydefault}{\sfdefault}    % tipo de fuente Arial
\changefontsizes{12 pt}                      % Tamaño de la fuente
\setlength{\parskip}{\baselineskip}          % Añade linea entre párrafos
\renewcommand{\baselinestretch}{1.5}         % Interlineado

% margen en 1.27cm
\makeatletter
\renewenvironment*{displayquote}
  {\begingroup\setlength{\leftmargini}{1.27cm}\csq@getcargs{\csq@bdquote{}{}}}
  {\csq@edquote\endgroup}
\makeatother

% Remueve la margen derecha de las citas textuales
\renewenvironment{quote}
{\list{}{}%
 \item\relax}
{\endlist}

\usepackage{hyphenat}          % Para enseñarle a dividir las palabras
\hyphenation{intera-cciones}
\hyphenation{CALINA}

% -----------------------------------------------------------------------------
% Configuración de títulos de secciones
% -----------------------------------------------------------------------------
\renewcommand{\thesection}{\arabic{section}}   % Remueve el número de capitulo en las secciones
\usepackage{sectsty}                           % Cambia fuente de los títulos de sección
\usepackage{titlesec}                          % Administración de títulos
\titlespacing{\section}{0pt}{0pt}{0pt}         % \titlespacing*{comando}{margen izquierdo}{espacio antes del titulo}{separación con el texto}
\titlespacing{\subsection}{0pt}{0pt}{0pt}
\titlespacing{\subsubsection}{0pt}{0pt}{0pt}
\titlelabel{\thetitle.\quad}                   % Añade un punto a los títulos

% -----------------------------------------------------------------------------
% Configuración de tabla de contenidos
% -----------------------------------------------------------------------------
\usepackage{tocloft} % Maneja entradas en el índice
\setlength\cftbeforesecskip{15pt}

\usepackage{chngcntr}                  % Las Numeración de figuras y tablas es continua
\counterwithout{figure}{chapter}
\counterwithout{table}{chapter}

\setcounter{tocdepth}{3}              % Sub sub secciones con numeración y en la tabla de contenidos
\setcounter{secnumdepth}{3}

% -----------------------------------------------------------------------------
% Configuración de referencias y citas
% -----------------------------------------------------------------------------
\usepackage[pdfencoding=auto,psdextra]{hyperref} % Enlaces, referencias
\hypersetup{ % Color y estilo de hiper-vínculos
	hidelinks,
	colorlinks=false,
} 
\usepackage[nottoc,notlof,notlot]{tocbibind}     % Para adicionar la bibliografía a la tabla de contenidos
\usepackage{apacite}
\bibliographystyle{apacite}

% -----------------------------------------------------------------------------
% Establece configuración manejo de imágenes
% -----------------------------------------------------------------------------
\usepackage{graphicx}                     % Propiedades extra para los gráficos
%\usepackage{graphbox}                    % Extensión para centrar gráficos
\usepackage[position=bottom]{subcaption}  % Permite usar sub-figuras
\usepackage{wrapfig}                      % Posición de imágenes, alrededor del texto
\graphicspath{{./img}}                    % Directorio por defecto

\usepackage{tikz}                % Para Hacer gráficos
\usetikzlibrary{positioning}
\usetikzlibrary{arrows}
\usetikzlibrary{scopes}
\usetikzlibrary{shapes}
\usetikzlibrary{calc}
\usetikzlibrary{patterns}
\usetikzlibrary{patterns.images} % Requiere descargar de https://raw.githubusercontent.com/Qrrbrbirlbel/pgf/master/tikzlibrarypatterns.images.code.tex
\usetikzlibrary{arrows.meta}
%\usetikzlibrary{fadings}        % Hace lo mismo que patterns.images para poner fondos en las formas
%ver https://tex.stackexchange.com/questions/566745/tikz-how-to-pass-an-image-by-a-filter
\usepackage{pgfplots}            % Creación de gráficos
\usepackage{pgf-pie}             % Creación de gráficos de pie
\usepackage{byo-twemojis}        % Para creación de emojis, requiere instalación manual
% Ejemplo de uso:
%\setlength{\twemojiDefaultHeight}{4em}
%\byoTwemoji{head; eyes normal; mouth laughing}

\usepackage{pgf-umlsd}           % Diagramas de secuencia UML
\usepackage{tikz-uml}            % Descargado de https://perso.ensta-paris.fr/~kielbasi/tikzuml/index.php
\tikzumlset{fill object = white, fill call = gray!20}

\usepackage{qrcode}

% -----------------------------------------------------------------------------
% Establece configuración encabezados y pie de páginas
% -----------------------------------------------------------------------------
\usepackage{fancyhdr}      % Encabezados y pie de páginas
\renewcommand{\sectionmark}[1]{\markboth{#1}{}}              % asignar \leftmark en sección en lugar de capitulo

% -----------------------------------------------------------------------------
% Establece configuración de listas
% -----------------------------------------------------------------------------
%\usepackage{paralist}                % Agrega funciones para el manejo de listas
\usepackage[inline]{enumitem}        % Permite mejorar enumeración ítems
\setlist{nosep}                      % Todas las listas sin separación
\setlist[itemize]{label=$\bullet$}   % Circulo relleno para las listas

% -----------------------------------------------------------------------------
% Establece configuración de Leyendas
% -----------------------------------------------------------------------------
\usepackage{caption}       % Leyendas
\captionsetup{labelfont=bf, justification=justified, singlelinecheck=false, labelsep = period}
\captionsetup[table]{skip=0pt}
\captionsetup[subfigure]{justification=centering}

% -----------------------------------------------------------------------------
% Establece configuración y librerías para tablas
% -----------------------------------------------------------------------------
\usepackage{array}         % Nuevas características a las tablas
\usepackage{bigstrut}      % permite mejorar unir filas en tablas y otras mejoras
\usepackage{longtable}     % Permite utilizar tablas en varias hojas
\usepackage{multirow}      % Agrega nuevas opciones a las tablas
\usepackage{booktabs}      % Tablas bonitas - revisar
\usepackage{tabularx}      % Otro entorno tabular
\usepackage{makecell}      % Mejora de celdas
%\usepackage{spreadtab}    % Para hacer hojas de cálculo

% -----------------------------------------------------------------------------
% Establece configuración y librerías manejo de páginas
% -----------------------------------------------------------------------------
\usepackage{pdflscape}     % Modo página horizontal de página

% -----------------------------------------------------------------------------
% Establece configuración y librerías manejo de pie de páginas
% -----------------------------------------------------------------------------
\usepackage[bottom,norule,hang]{footmisc}

% -----------------------------------------------------------------------------
% Establece configuración y librerías manejo de apéndices y glosario
% -----------------------------------------------------------------------------
\usepackage[toc]{appendix}
\usepackage[acronym]{glossaries-extra}

% -----------------------------------------------------------------------------
% OTRAS
% -----------------------------------------------------------------------------
\usepackage[binary-units]{siunitx}	%Manejo de unidades de medida
\usepackage{algpseudocode}              %Para pseudocódigo
\usepackage{listings}			%Para código fuente
\usepackage{amsthm}			%Para hacer definiciones y teoremas en matemáticas
\usepackage{tcolorbox}			%Cajas de texto de colores
\usepackage{dialogue}			%Hacer diálogos entre personas
%\usepackage{awesomebox}		%cajas de texto con iconos, incompatible con fontawesome
%\usepackage{fancypar}                  %Párrafos personalizados
%\usepackage{lineno}			%Numerar lineas del documento
\usepackage{comment}			%para comentar muchas lineas

% -----------------------------------------------------------------------------
% LIBRERÍAS PDF
% -----------------------------------------------------------------------------
\usepackage{pdfpages}      % Insertar pdfs en el documento
\usepackage{hyperxmp}      % Etiquetas opcionales para el pdf compilado

\hypersetup{%
	pdftitle={\titulotesis},
	pdfauthor={\autordeldocumento},
	pdflang={es},
	pdfkeywords={\palabrasclave},
	pdfpublisher={\nombreuniversidad}
}

