% Autor: Jose Ricardo Bustos Molina
%        Universidad del Tolima
%        jrbustosm@ut.edu.co
%

\chapter{Instrumentos para recolección de datos usados en del desarrollo de la investigación cualitativa}
\label{anexo:encuestas}

\clearpage % Salto de página
\section{Entrevista no estructurada inicial}

\begin{dialogue}
	\speak{Ricardo} Buenas noches profesora Gloriset, quería saber que sabes sobre gamificación, lo has 
	usado en tus clases?
	\speak{Docente Gloriset} Gamificación, referido a utilizar juegos en el aula o ambiente de aprendizaje 
	y que tienen como objetivo la enseñanza y aprendizaje de determinado tema o habilidad
	\speak{Ricardo} y lo has usado en clase?
 	\speak{Docente Gloriset} si, suelo usar juegos de memoria, vídeo lecciones, juegos de mesa (bingo, 
	serpiente)
 	\speak{Ricardo} Muy bien, desde su perspectiva es indispensable usar un juego para aplicar 
	gamificación? Los estudiantes siempre deben jugar para tener una clase gamificada?
	\speak{Docente Gloriset} No considero que sea jugar como tal sino practicar en clase o aprender 
	mediante actividades lúdicas, se hace necesario que el estudiante tenga la oportunidad de interactuar 
	de manera lúdica con el conocimiento para que le sea mas fácil su comprensión.
	\speak{Ricardo} Desde su experiencia, una estrategia sencilla como poner un sello en el cuaderno ``por 
	ejemplo'' ante una buen comportamiento por parte del estudiante?  Lo puedes considerar un juego o un 
	premio?
	 \speak{Docente Gloriset} Para mi considero se vuelve un incentivo por su participación en clase no 
	 por un buen comportamiento, que puede al final del periodo contribuir a mejorar la nota final.
	 \speak{Ricardo} Considera usted que estos incentivos como puntos extras, sellos, incentivan al 
	 estudiante a terminar sus actividades a tiempo o de mejor manera, cómo sus estudiantes reaccionan al 
	 promover estos?
	 \speak{Docente Gloriset} Incentivan a culminar las actividades, a participar aunque se equivoquen ya 
	 que no solo lo reciben el sello por que este bien la respuesta. Muchos buscan de cierta manera 
	 recibir su sello; algunos lo hacen porque requieren al final de corte subir su promedio, otros por 
	 validar que su trabajo este bien y que hayan comprendido.
	 \speak{Ricardo} desde su ejercicio profesional, la entrega de incentivos se debe hacer siempre que un 
	 estudiante haga algo bien?, ha considerado usted entregar incentivos en otros contextos, por ejemplo 
	 al equivocarse en el desarrollo de un ejercicio?
	 \speak{Docente Gloriset} Cómo indique antes no solo es si el estudiante hizo algo bien, es por el 
	 hecho de participar de la clase o realizar alguna tarea propuesta y socializarla en clase. En otros 
	 contextos, ahora que los jóvenes se preparan para el \textit{English day}, los uso para seleccionar 
	 los competidores de un concurso de deletreo que se tiene pensado saber los finalistas por ejemplo.
	 \speak{Ricardo} Para los incentivos que usted me menciona promueve en los estudiantes sentimientos de 
	 valoración, reconocimiento, realización y competición, sin embargo deseo saber si en el ejercicio de 
	 sus clases promueve sentimientos de altruismo (ayudar a las personas) o de exploración 
	 (descubrimiento)?
	 \speak{Docente Gloriset} No, en el momento solo lo uso para lo que tiene que ver con lo académico.
	 \speak{Ricardo} Para el desarrollo de su guía de aprendizaje, tiene usted en cuenta los gustos u 
	 objetivos de sus estudiantes?
	 \speak{Docente Gloriset} Los gustos y sobre todo las  habilidades (escucha, habla, lectura, 
	 escritura) que se fortalecen desde la enseñanza del inglés.
	 \speak{Ricardo} Podrías decirme un ejemplo sobre esto que me dices?
	 \speak{Docente Gloriset} Busco actividades con canciones en Inglés, canciones con algún artista que 
	 les guste y este acorde al tema a trabajar, vídeo lecciones con personajes animados o de películas 
	 que les gusten para fortalecer la escritura o vocabulario, pictionary para lo que respecta al habla o 
	 role plays.
	 \speak{Ricardo} En la guía de aprendizaje que realiza usted, realiza de antemano una planificación de 
	 los incentivos y puntos extras, o estos van surgiendo en el desarrollo de las clases?
	 \speak{Docente Gloriset} Va surgiendo en el desarrollo de la clase
	 \speak{Ricardo} Existe un significado adicional estos puntos extras o incentivos, entre sus 
	 estudiantes? Osea, a parte de significar un aumento de nota, existe algo adicional que me puedas 
	 contar?
	 \speak{Docente Gloriset} Quizás que para ellos sea la clase un espacio donde puedes participar así tu 
	 respuesta sea o  o acertada , un espacio para aprender ya que el miedo de ellos es muchas veces 
	 hablar y pronunciar mal o no saber si tiene sentido lo que dicen y quedar en ridículo. La idea es que 
	 finalmente les quede claro el tema a todos y dejar de lado y sea un espacio para resolver cualquier 
	 error o inquietud o fortalecer las habilidades que ya traen.
	 \speak{Ricardo} ¿Cuál cree usted, es la función del profesor en una clase gamificada?
	 \speak{Docente Gloriset} Formar al estudiante en determinadas competencias, que fortalezca las 
	 habilidades del estudiante.
	 \speak{Ricardo} y ¿Cuál cree usted, es la función del estudiante en una clase gamificada?
	 \speak{Docente Gloriset} Que aprenda, que interactúe con el tema , en lo social-relacional que 
	 interactúe con sus compañeros.
	 \speak{Ricardo} Desde su perspectiva, que riesgos tiene el usar retos, incentivos o puntos extras en 
	 los estudiantes?
	 \speak{Docente Gloriset} que no sean atractivos para el estudiante al punto en que no participen o no 
	 les ayude a aprender, que este (os) no los aproveche y se pierda el objetivo de la clase .
	 \speak{Ricardo} Volviendo a lo que me había dicho antes, sobre como otorga puntos extras o promueve 
	 competencias de deletreo, me puede describir detalladamente el proceso de asignación de estos 
	 incentivos?
	 \speak{Docente Gloriset} se organizan los estudiantes por filas, a cada una se le muestra una imagen 
	 y ellos de manera ordenada deben levantar la mano . Se le da la palabra al primero que lo hizo y todo 
	 el grupo escucha atentamente , si el estudiante deletreo bien allí obtiene el punto . Se hacen 2 o 
	 tres rondas y se va tomando al  o los ganadores de la sesión  1 por cada fila. Se van a hacer 10 
	 sesiones de práctica y la idea es que ya para le proceso de selección los que se postulen puedan ya 
	 competir como tal. Los puntos recogidos durante las semanas de práctica se tendrán en cuenta para la 
	 co-evaluación y su nota final
\end{dialogue}

\clearpage % Salto de página
\section{Entrevista no estructurada final}

\begin{dialogue}
	\speak{Ricardo} Usted cree que como se implementó CALINA en su curso le añadió un trabajo extra a 
	usted? Le facilitó o dificultó procesos de su desarrollo docente?
	\speak{Docente Gloriset} Sí, se requiere un trabajo extra pero en cuanto a planear como se va a hacer 
	uso de la estrategia en clase, por supuesto facilita el proceso en cuanto a que puede lograr que el 
	estudiante se motive a participar.
	\speak{Ricardo} Usted siente que la estrategia CALINA motiva a algunos estudiantes?
	\speak{Docente Gloriset} Hay jóvenes como en todo grupo que son competitivos, les gusta preguntar, 
	aprender y mejorar sus procesos por lo cual esta estrategia si considero motiva al estudiante.
	\speak{Ricardo} y por otro lado, Usted siente que la estrategia CALINA, desmotiva a algunos 
	estudiantes?
	\speak{Docente Gloriset} Podría desmotivar si el estudiante cree que no tiene o no puede  hacer parte 
	de las actividades o obtener incentivos por no contar con la app. Ciertamente podría  para ellos es 
	mas emocionante intercambiar cartas, obtener medallas, dragones etc, que un sello en un cuaderno.
	\speak{Ricardo} Podrías describirme de estos sentimientos cuales cree usted se manifiestan en sus 
	estudiantes con la estrategia usada: Valoración, Reconocimiento, Realización, Individualización, 
	Competición, Altruismo, Socialización, Exploración, Desafío, Control, Fantasía, Sufrimiento?
	\speak{Docente Gloriset} Reconocimiento, realización, competición, socialización, exploración y 
	Fantasía. No considero que genere sentimientos negativos ya que se trata de mejorar, de incentivar.
	\speak{Ricardo} Me podrías dar un ejemplo sobre como se manifiestan algunos de estos sentimientos 
	desde su perspectiva?
	\speak{Docente Gloriset} Reconocimiento al sentir que sus compañeros  lo reconocen por sus 
	habilidades. Socialización al interactuar dentro de las actividades propuestas para lograr que se 
	cumpla la estrategia o se obtengan los incentivos.
	\speak{Docente Gloriset} Fantasía, ya que la misma estrategia propone historias para recibir cartas 
	con determinados personajes. Esto lleva a que el estudiante se involucre.
	\speak{Ricardo} Muy bien, y en que ayuda la estrategia CALINA respecto al no uso de esta, las 
	actividades se desarrollan más rápido, más completas, de mejor calidad, o solo es un cambio en el 
	estado de animo, o incluso pasa lo contrario que las actividades no se entregan o se baja la calidad 
	de estas?
	\speak{Docente Gloriset} No la entendí?
	\speak{Ricardo} En otras palabras usted cree que aplicando la estrategia CALINA, mejora el rendimiento 
	escolar, o mejora la disposición de los estudiantes respecto a sus actividades?
	\speak{Docente Gloriset} Mejora el rendimiento escolar ya que entre más participación del estudiante 
	en clase que se traduce en uso de la estrategia, hay una mayor comprensión de los temas o competencias 
	tratadas en clase.
	\speak{Docente Gloriset} En cuanto a disposición, como son jóvenes si varía dependiendo a veces del 
	estado de ánimo de ellos. Están en una edad en la que así como hoy llegan con ánimo y trabajan hay 
	días en los cuales no están dispuestos a participar en nada y se requiere cambiar de estrategias.
	\speak{Ricardo} Podría decirme cual estrategia le gusto mas construir y trabajar?, la que uso los 
	símbolos precolombinos o la que usa la historia creada de los dioses? En los estudiantes usted cree 
	que es relevante contar una historia?
	\speak{Docente Gloriset} Ambas funcionan bien, en la primera algunos indagaron sobre el significado de 
	los símbolos y de las cartas, en la segunda la historia les causó curiosidad por saber de dónde surgió 
	la historia y los hizo fantasear o sentirse parte de ella  y a otros  diversión.
	\speak{Ricardo} Usted pudo ver dinámicas entre los grupos (se hablan entre los novenos), o entre los 
	estudiantes comportamientos diferentes a los esperados?
	\speak{Docente Gloriset} El tiempo para su aplicación en clase a sido corto por lo que entre los 
	grupos no he identificado alguna interacción que involucre la estrategia sin embargo dentro de la 
	clase si entre los jóvenes, están hablando de la estrategias, miran con curiosidad y exploran la 
	aplicación y que se puede hacer con ella lo que ha llevado a que socializen aun más entre ellos.
	\speak{Ricardo} Por último me puedes describir que aspectos positivos tiene la aplicación CALINA, y en 
	que podría mejorarse?
	\speak{Docente Gloriset} Aspectos positivos: la aplicación tiene un diseño muy hermoso, los colores 
	son muy del Tolima y  para los estudiantes es muy intuitivo, fácil de entender y usar. En cuanto a 
	mejorar, se requiere mejorar o hacer más fácil aún la manera de crear las cartas ya que la opción de 
	opciones avanzadas ``trigger'' requiere ser un poco más especializado para crear una carta con cosas 
	especiales. Se trata entonces de que el docente deba leer mejor el manual para poder optimizar el uso 
	de la app en clase quizás es algo que no todos vayan a hacer y por lo tanto pueden desechar su uso.
\end{dialogue}
